\section{Příklad 2}
% Jako parametr zadejte skupinu (A-H)
\druhyZadani{D}


\begin{figure}[h!]
    \begin{circuitikz} \draw
    (0,0) -- (0,4)
    (0,4) to[R ,l_=$R_1$](3,4)
    (3,4) to[R , -* ,l_=$R_3$](3,2)
    (3,2) to[R , -* ,l_=$R_4$](3,0)
    (3,2) -- (5,2)
    (5,2) to[R , -* ,l_=$R_5$](5,0)
    (3,0) -- (5,0)
    (0,0) to[R ,l_=$R_2$](3,0)
    ;
    
    \end{circuitikz}
    \centering
    \caption{Zkratovaní zdroje napětí}
\end{figure}

\begin{figure}[h!]
    \begin{circuitikz} \draw
    (0,0) -- (0,4)
    (0,4) -- (3,4)
    (3,4) to[R , -* ,l_=$R_{123}$](3,2)
    (3,2) to[R , -* ,l_=$R_4$](3,0)
    (3,2) -- (5,2)
    (5,2) to[R , -* ,l_=$R_5$](5,0)
    (3,0) -- (5,0)
    (0,0) -- (3,0)
    ;
    
    \end{circuitikz}
    \centering
    \caption{Zjednodušení $R_1$, $R_3$ a $R_2$}

    \begin{gather*}
        R_{123} = R_{1} + R_{2} + R_{3}  \\
        R_{123} = 200 + 200 + 660\\
        R_{123} = 1060 \: \si\ohm
    \end{gather*}

\end{figure}

\begin{figure}[h!]
    \begin{circuitikz} \draw

    (0,0) -- (0,4)
    (0,4) -- (3,4)
    (3,4) to[R , -* ,l_=$R_{123}$](3,2)
    (3,2) to[R , -* ,l_=$R_4$](3,0)
    (3,2) to[short, -o] (5,2)
    node[label={[font=\footnotesize]above:A}] {}
    (3,0) to[short, -o] (5,0)
    node[label={[font=\footnotesize]above:B}] {}
    (0,0) -- (3,0)
    ;
    
    \end{circuitikz}
    \centering
    \caption{Nahrazení $R_5$}
\end{figure}


\begin{figure}[h!]
    \begin{circuitikz} \draw
    
    
    (2,1) to[short, -o] (0,1)
    node[label={[font=\footnotesize]above:A}] {}
    (2,1) to[short, *-] (2,2)
    (2,2) to[R ,l_=$R_{123}$](4,2)
    (4,2) to[short, -*] (4,1)
    (2,1) to[short, *-] (2,0)
    (2,0) to[R ,l_=$R_{4}$](4,0)
    (4,0) to[short, -*] (4,1)
    (4,1) to[short, -o] (6,1)
    node[label={[font=\footnotesize]above:B}] {}
    ;
    
    \end{circuitikz}
    \centering
    \caption{$R_{123}$ a $R_4$ paralelně}
\end{figure}

\begin{figure}[h!]
    \subsection{Výpočet $R_i$ a $U_i$}

    \begin{gather*}
        R_{i} = \frac{R_{123} \times R_{4}}{R_{123} + R_{4}} \\
        R_{i} = \frac{1060 \times 200}{1060 + 200} \\
        R_{i} \doteq 168.254 \: \si\ohm \\
    \end{gather*}

    \begin{gather*}
        U_i = U \times \frac{R_4}{R_1+R_2+R_3+R_4} \\
        U_i = 150 \times \frac{200}{200+200+660+200} \\
        U_i \doteq 23.8095 \: V 
    \end{gather*}
\end{figure}

\begin{figure}[h!]

    \subsection{Výpočet $I_{R5}$ a $U_{R5}$}

    \begin{gather*}
        I_{R5} = \frac{U_i}{R_i+R_5} \\
        I_{R5} = \frac{23.8095}{168.254 + 550} \\
        I_{R5} = 0.0331 \: A
    \end{gather*}
    \begin{gather*}
        U_{R5} = R_5 \times I_{R5} \\
        U_{R5} = 550 \times 0.0331 \\
        U_{R5} = 18.205 \: V
    \end{gather*}
\end{figure}

\clearpage